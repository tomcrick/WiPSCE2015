% This is ''sig-alternate.tex'' V2.0 May 2012
% This file should be compiled with V2.5 of '\'sig-alternate.cls'' May 2012
%
% This example file demonstrates the use of the \'sig-alternate.cls'
% V2.5 LaTeX2e document class file. It is for those submitting
% articles to ACM Conference Proceedings WHO DO NOT WISH TO
% STRICTLY ADHERE TO THE SIGS (PUBS-BOARD-ENDORSED) STYLE.
% The \'sig-alternate.cls' file will produce a similar-looking,
% albeit, 'tighter' paper resulting in, invariably, fewer pages.

\documentclass{sig-alternate}
\sloppy
\usepackage{paralist}
\usepackage{url}

\begin{document}
%
% --- Author Metadata here ---
\conferenceinfo{WIPSCE}{2015 London, UK}
\CopyrightYear{2015} % Allows default copyright year (20XX) to be over-ridden - IF NEED BE.
%\crdata{0-12345-67-8/90/01}  % Allows default copyright data (0-89791-88-6/97/05) to be over-ridden - IF NEED BE.
% --- End of Author Metadata ---

\title{Technocamps: A Decade of Supporting\\Computer Science Education in Wales}

\numberofauthors{2}
\author{
% 1st. author
\alignauthor
Tom Crick\\
\affaddr{Department of Computing}\\
\affaddr{Cardiff Metropolitan University, UK}\\
\affaddr{tcrick@cardiffmet.ac.uk}
% 2nd. author
\alignauthor
Faron Moller\\
\affaddr{Department of Computer Science}\\
\affaddr{Swansea University, UK}\\
\affaddr{F.G.Moller@swansea.ac.uk}\\
}

\maketitle

\begin{abstract}
Computing education in the UK over the past five years has undergone
significant scrutiny and upheaval. In September 2014, we saw the
implementation of a new computing curriculum in England, alongside
significant investment in the professional development of teachers in
Scotland. However, in Wales, numerous political and geographical
issues have hindered any substantive educational policy or curriculum
reform for computing education. This is despite the fact that Wales
was addressing the failings of computing education in schools since at
least 2003 through Technocamps, a pan-Wales university-based schools
outreach programme. In this paper we outline the history (and
pre-history) of Technocamps; explain the devolved nature of education
in the UK focusing on Wales with its specific issues and challenges;
and present data both in support of university engagement and
intervention as well as the positive effect this intervention is
having.
\end{abstract}

% A category with the (minimum) three required fields
\category{K.3.2}{Computers \& Education}{Computer and Information Science Education}[Computer Science Education]
\category{K.4.1}{Computers And Society}{Public Policy Issues}
\keywords{Computer Science Education; High School; Teachers}

% TO DO: premable -- a note on terminology? e.g. ICT, etc?

\section{Introduction}
% taken from TOCE pitch from 2013
In the early 1980s, the BBC Micro was introduced to schools throughout
Britain for the \emph{BBC Computer Literacy Project}.
Before long they were in 80\% of UK classrooms~\cite{vasko:1986}.
By encouraging young learners to experiment with computers, a generation
of creative (and computational) talent was spawned. Applications in
the UK to study computer science at university hit a peak, and
computer science graduates changed the world as they helped computers
come to dominate every aspect of our lives.

Fast forward 30 years and the situation could not be any more
different. The computer is no longer a novelty. Children now typically
spend more time at home in front of a computer screen than a TV screen, but
like the TV, their interest is restricted to using the computer, not
in experimenting with it. Computer studies in school -- now called
Information and Communication Technology (ICT) -- has evolved into IT
studies with an emphasis on digital literacy and office skills --
significantly more mundane than the social networking and gaming for
which the pupils use their home computers. A full 66\% of ICT teachers
in the UK do not have a relevant qualification but have slipped into
the role of ICT teacher simply by being sufficiently digitally
literate~\cite{RoyalSoc:2012}.
The situation is worse in Wales, where this figure rises to
75\%~\cite{GTCW:2008}.
Applications to
study computer science at university slumped in the early part of the
millennium -- especially amongst females -- and
many of those who started a university computer science degree course
found themselves dropping out during the first year as they entered
unaware of what computer science is and what studying it entails.

In the early 2000s, the Department of Computer Science at Swansea University
started looking into ways to address this issue.
Unfortunately, attempts to reach out to teachers in local schools
faced great resistance, due naturally to their lack of confidence
in anything more complicated than using a desktop package.

As an alternative route into schools, Swansea University created
\emph{Technocamps} in 2003, an outreach programme which brings groups
of school children to the University campus for day-long workshops based on
selected computational themes to inform them what computing is about,
followed-up by support in setting up
extracurricular clubs -- \emph{Technoclubs} -- in the schools.
Technocamps proved very successful as a local initiative, with many
students studying computer science at Swansea University claiming to be
influenced by Technocamps activities.

In 2010, based on empirical data
regarding its effect on school children's attitudes towards computing
-- as well as their teachers -- Swansea University was awarded
\pounds 3.9 million funding towards a \pounds 6 million four-year project
(with the remaining \pounds 2.1 million
generated through University match funding)
by the Welsh Government under the EU's European Social
Fund (ESF) Convergence Programme to run Technocamps as a pan-Wales
project with regional hubs at
the Universities of Aberystwyth, Bangor and
South Wales Glamorgan.
(Technocamps hubs have subsequently been set up at most of the remaining
major Universities in Wales, specifically Cardiff University,
Cardiff Metropolitan University, and Glynd\^wr University in Wrexham.)
Though focusing on the children, Technocamps also provides ``Technoteach''
events aimed at up-skilling ICT teachers in Welsh schools.
Technocamps has since provided computing-related activities and resources
for tens of thousands of young people across Wales, as well as interacting
with hundreds of teachers at hundreds of the nation's schools.

Technocamps is not alone in exploring solutions to a
perceived problem in Computer Science education.
In particular, in 2008 the
Computing At School (CAS) organisation was formed, and its current
membership of over 18,000 teachers and computing professionals
work hard to promote the teaching of computing at school in England.
However, whilst great changes have taken place in England
due in no small part to CAS lobbying
-- underpinned by generous funding of CAS by
the English Department of Education --
the CAS effect is relatively unnoticed in Wales,
and the rapid changes pushed through in England
are in many ways resisted by Welsh Government.

Wales is a devolved nation within the United Kingdom, with its own
elected national government fully responsible for its education system.
At the 2012 Annual Technocamps Teachers' Conference,
the Welsh Government's Minister for Education and Skills
publicly acknowledged the importance of computer science education;
noted that he is a key supporter of Technocamps;
and expressed understanding of the wider educational and
socio-economic impact that the government can make with reform in Wales.
However, with only 5\% of the population of England and with its distinct
geographical and socio-cultural challenges, Wales presents
a variety of unique challenges in addressing curriculum reform.

In this paper, we will describe the backdrop to Technocamps and why it
was created in the way it was (Section 2); explain the devolved nature
of education in the UK focusing on Wales with its specific challenges
and issues (Section 3); and present data both in support of university
intervention as well as the positive effect this intervention is
having (Section 4).  We finish with a consideration of the challenges
remaining in Welsh education (Section 5).

\section{Computing Education in Wales and England}

In the 1980s, Computer Studies was a popular subject
in schools across Britain. The ubiquitous presence, in schools and homes,
of the popular BBC Micro -- which was useful for little else than
writing programs in BASIC -- saw a large proportion
of school children learning the fundamentals of program design
in a curriculum which included a variety of complementary
topics such as hardware, software, Boolean logic
and binary number representation~\cite{Doyle:1988}.

By the 1990s, however, the emergence of pre-installed software
packages -- specifically word processors and spreadsheet programs --
meant that computers were no longer predominantly machines that needed
to be programmed in order to do anything useful or interesting.  Less
and less time was being spent in the Computer Studies classroom on
thinking about and writing programs, as basic digital literacy became
regarded as its key skill.  However, as interest in viewing the
computer as a creative tool waned in favour of using it for more
mundane tasks, various problems were being created, which were
highlighted in two independent enquiries in 1997: the McKinsey
Report~\cite{McKinsey:1997} and the Stevenson
Report~\cite{Stevenson:1997}.  Both reports concluded that Information
Technology in UK schools was in a primitive state and in need of
attention and major investment.  In line with the Stevenson
Report, Computer Studies evolved into a new
subject whose name was coined in that Report: Information and
Communication Technology (ICT).  Over the decade starting in 1997, the
UK Government invested over \pounds3.5 billion in ICT in schools
through various initiatives such as the National Grid for Learning
(NGfL) and the New Opportunities Fund (NOF)~\cite{Doughty:2006}.

By 2000, then, ICT had permeated both primary and secondary school
curricula.  The emphasis was on developing the children's IT skills
and digital literacy in an honest attempt to address the increasing
need for digital literacy amongst the public.  However, despite
enormous government-funded ICT initiatives, various reports throughout
the decade identified problems with implementing government policy on
ICT educational
reform~\cite{OpieFukuyo:2000,Ofsted:2001,Ofsted:2002,Ofsted:2004,
Loveless:2005}.  Younie~\cite{Younie:2006} summarises the problems
identified by these reports into five key areas, three being
management and the other two being: teacher training and competence;
and impact on pedagogy.

A decade later, a report by the Royal Society~\cite{RoyalSoc:2012}
makes the very same observations.  The report noted that ICT suffers
from a poor reputation amongst pupils and other stakeholders, who
consider it dull and unchallenging and hence a low-value discipline,
especially compared to other STEM subjects.  With ICT embedded across
the primary school curriculum, secondary school pupils find ICT
in secondary school neither stimulating nor engaging.  The Wolf
report~\cite{Wolf:2011} further notes that the undemanding nature of
ICT qualifications in secondary schools is readily exploited by the
schools: due to a high league table weighting associated with
vocational qualifications, easily-achieved high results in ICT offer a
welcome boost to a school's league table position.  Furthermore, as
ICT is typically presented by schools as their computer science
offering, students who might otherwise enjoy studying computer science
are left with a positive distaste for what they are led to believe is
computer science.

\subsection{Technocamps at Swansea University}
As experienced by other Universities,
the numbers of students enrolling in Computer Science
at Swansea University increased through the end
of the millennium due to the dot-com boom.
However, as depicted in Figure~\ref{fig:numbers},
\begin{figure}
  \centering
  \includegraphics[width=0.9\columnwidth]{images/numbers.png}
  \caption{Applications to University Computer Science programmes
           in the UK (top) and Wales (bottom), males (blue) and females (pink).
           (Source: UCAS, Universities and Colleges Admissions Service,
            https://www.ucas.com)}
  \label{fig:numbers}
\end{figure}
throughout the UK (as elsewhere) the numbers then peaked, and what followed
was a steady five-year decline, dropping more than 40\% during that time,
with the worst effect on the already-dwindling numbers
of female students.
Even at its peak, more than a third of students who
started a computer science degree programme left
the programme before their second year of study,
citing a mistaken understanding of the subject
as their reason for leaving.

In an attempt to address this anomaly,
the Computer Science Department at Swansea reached out
to local secondary school ICT teachers,
inviting them to meetings at the University,
and offering to visit schools to discuss
the subject with the teachers and to give
motivational talks to students.
Indeed, the Department was invited every year to
a number of English schools to present such talks
to school children making University choices.
However, interest locally was more than absent:
there was positive resistance to the Department
giving talks to their University seekers;
such activity was typically characterised as merely ``pitching for students.''
In reality, for reasons explained later
which did not apply to English teachers,
teachers in Wales were generally feeling over-burdened and
uninterested in exploring any perceptions
of inadequacy in their curriculum delivery.

As it proved impossible to influence schools and their ICT teachers directly,
Technocamps was created in 2003 to promote computing amongst their pupils.
This was a programme of fun and engaging interactive computational workshops
taking place on the University campus
whose ultimate aim was to re-introduce computer science into
the ICT curriculum by generating the demand from the students.
Originally run only at Swansea University,
Technocamps hubs have since been created at most major Universities
throughout Wales.

Welsh teachers were happy to ``treat'' their classes
to these ``day out'' activities; but they were then faced with
the prospect of satisfying their pupils' newly-discovered passion
by introducing ``Technoclubs'' as lunch-time
extra-curricular activities in the school.
With generous help, resources and guidance from Technocamps
-- along with the fact that the students were generally
more technically informed and computer-literate than their teachers -- 
these clubs have flourished, and the impact of Technocamps
in changing attitudes in Welsh schools regarding ICT and computing
has been well acknowledged.
An independent Review~\cite{Wavehill:2015}
of Technocamps activity in the so-called convergence
(ie, economically disadvantaged) region of Wales
(see Figure~\ref{fig:wales})
\begin{figure}
  \centering
  \includegraphics[width=0.45\columnwidth]{images/UK.png}
  \caption{Map of England and Wales}
  \label{fig:UK}
\end{figure}
\begin{figure}
  \centering
  \includegraphics[width=0.45\columnwidth]{images/wales.png}
  \caption{Convergence region of Wales, encompassing 15 out of 22 local councils}
  \label{fig:wales}
\end{figure}
carried out for Welsh Government estimates that
5\% of Welsh secondary students (ie, aged 11-19)
have engaged with Technocamps through Workshops,
and that more than a quarter of the secondary schools
in the region have established Technoclubs.

% taken from TOCE paper, can be adapted to Welsh focus
\section{Education in the UK}\label{sec:schools}

The UK consists of four nations ruled by one parliament:
England (population: 53.0 million), Scotland (5.3 million),
Wales (3.0 million) and Northern Ireland (1.8
million)\footnote{\url{http://www.ons.gov.uk/ons/guide-method/census/2011/index.html}}.
In 1997, Scotland and Wales held referendums which
determined in both cases the desire for self-government.
In the case of Wales, this led to the Government of Wales Act 1998
which created the National Assembly for Wales, to which
a variety of powers were devolved from the UK parliament
on 1 July 1999.
In particular, education
-- which until then was a UK-wide government portfolio --
came under the control of the National Assembly for Wales,
under the direction of the Department for Education and Skills.

% % change this to have a brief discussion about all of the main
% systems in UK, especially England and Scotland (imp: Curric. for
% Excellence, plus Computing Science), then focus on Wales...

Wales is a small nation to the west of England (see Figure~\ref{fig:UK}).
It has an ancient culture and a thriving language.
Its South coast became pre-eminent in coal mining and heavy industry;
however, it is mostly rural and suffers from rural poverty,
seasonal employment and the collapse of industry.
The country is sparsely populated with few top-quality roads.
Hence its communities
-- and more specifically its schools and teachers --
suffer from the perils of isolation.
Apart from the south east corner (including its capital Cardiff)
and the regions bordering England, the whole country is
formally designated within the European Union (EU) as so-called
``convergence area,'' meaning its per-capita gross domestic product (GDP)
is less than 75\% of the EU average.

Prior to devolution, the education system in Wales was essentially
identical to that in England and was in a very healthy state,
outperforming all other regions in the UK
in the years prior to and immediately following devolution.
However, since devolution it has suffered a rapid decline.
Evans~\cite{Evans:2015} carries out a systematic and detailed
analysis as to why this was the case.

Whilst maintaining the general educational system of Key Stages
used in England (See Figure~\ref{fig:key-stages}),
\begin{figure}
  \centering
  \includegraphics[width=\columnwidth]{images/keystages.png}
  \caption{Key Stages in the English and Welsh education system}
  \label{fig:key-stages}
\end{figure}
the Welsh Government embarked on a 10-year revolutionary plan
including
\begin{itemize}
\item
phasing out Standard Attainment Tests (SATs);
\item
replacing the early Key Stage 1 with
a play-based Foundation Phase;
\item
introducing the Welsh Baccalaureate at all levels:
an overarching qualification,
with a purely practical-based assessment mechanism,
incorporating
key skills; Wales, Europe and the world;
work-related education; and personal and social education;
\item
emphasising the focus on the Welsh language and  Welsh-medium schools;
and
\item
addressing the abundance of small schools in the 
predominantly rural communities throughout Wales.
\item
tackling deprivation.
\end{itemize}
Much of this plan was lauded, and time may yet prove its merits.
However, its implementation has been criticised
for various reasons and by various stakeholders.
The Minister for Education newly-appointed in June 2010,
in looking for the causes of Wales' failing education system,
found cause to commission
no fewer than 24 costly reviews
before his untimely resignation in February 2013 -- almost one per month.

With devolved autonomy comes autonomy over fiscal matters;
and the correlation between money and performance is
an obvious target for critics, who point to a growing spending shortfall 
between Wales and England.
The average spend per pupil in Wales in 2000-2001
-- just after devolution --
was more than every region of England apart from
the large metropolitan areas of London, the West Midlands and the North West,
all of which benefit from their vast economies of scale.
However, since then, the gap between
the education budgets per pupil between Wales and England
has steadily grown by about 1\% per year;
the figures forecast for 2013-2014 show
13\% more being spent per pupil in England than in Wales
(\pounds7,533 per pupil in England as opposed to
\pounds6,676 per pupil in Wales).

%As reported by Hubweiser et al.~\cite{hubwieser-et-al:2011}, when establishing
%a model for viewing school CS education, it is apparent that there is
%much diversity between school education systems, and this can create
%an obstacle when trying to understand progress made in a different
%country. Here we describe the context of school education in the UK.

% also: quals reform -- talk about WJEC and Wales in context of GCSEs
% and A-Levels!

%compulsory schooling until age 16.
%All subjects are
%compulsory until the end of Key Stage 3 (KS3) and then students can
%choose approximately ten subjects to study for the next two years,
%which each lead to GCSE (General Certificate of Secondary Education)
%qualifications. However, while the National Curriculum in England and
%Wales are broadly similar, they are distinct and use different
%terminology.

% adjust to be Wales-focused
%There is state provision for education in the UK up to the age of 19,
%with mostly comprehensive, mixed ability schools across the UK. A few
%areas in England have retained a system of selective 11+ schools
%called grammar schools, which require students to sit an exam prior to
%entry, but these schools are in the minority. As well as state
%schools, 10\% of schools in the UK are independent fee-paying
%schools. Overall, in England there are approximately 24,000 schools,
%including 16,800 primary schools, 3,400 secondary schools and 2,400
%independent schools (primary and secondary).  However, the primary and
%independent schools tend to be smaller: the state-funded schools had
%4.2 million primary pupils and 3.2 million secondary pupils, with 0.6
%million pupils in independent schools.

%The ICT curriculum in Wales
%(2008)\footnote{\url{http://wales.gov.uk/topics/educationandskills/schoolshome/curriculuminwales/arevisedcurriculumforwales/nationalcurriculum/ictnc/?lang=en}},
%was perceived to be less prescriptive than the ICT curriculum in
%England, but exhibiting many of the same issues. It was recently
%reviewed by an independent steering group appointed by the Welsh
%Government~\cite{welshictreview:2013}, making clear recommendations for
%reforming the ICT curriculum as part of a broader national curriculum
%review for September 2014.
%
% update end of this section so that it leads into the Donaldson review...

\subsection{2013 ICT Curriculum Review}

% \textbf{Tom -- this is your story~\cite{welshictreview:2013};
% Please write it.}

In light of significant reform since devolution of education policy to
Wales in 1999, there has been much focus on the use of technology in
education. In September 2011, the Minister for Education and Skills
commissioned a review of ``digital classroom teaching'', setting up an
independent group to identify which digital classroom delivery aspects
should be adopted to transform learning and teaching for those aged 3
to 19. In particularl this review focused on how e-infrastructural
issues, such as high-quality, accessible digital classroom content
could be developed, but specifically on how teachers might get the
digital teaching skills to use ICT to transform schools. Their
report~\cite{haywarddigwales:2012}, published in March 2012,
highlighted how digital technologies, combined with sound pedagogy,
can make a substantial difference to learning experiences and
performance, recommends actions to introduce, embed and promote the
use of digital technologies in
education\footnote{\url{http://gov.wales/topics/educationandskills/publications/wagreviews/digital/?lang=en}}. Alongside
the commitment of significant funding for e-infrastructure to support
learning and teaching in Wales, in September 2012 the Welsh Government
established the National Digital Learning Council\footnote{N.B. Crick
is a member of the NDLC (2012-present), with Moller a special
advisor.}\footnote{\url{http://gov.wales/about/cabinet/cabinetstatements/2012/learningindigitalwales/?lang=en}}
to provide expert and strategic guidance on the use of digital
technology in teaching and learning in Wales.  The remit of the
Council was to guide the implementation of the {\emph{Learning in
Digital Wales}} programme (a strategic investment on next-generation
connectivity for schools in
Wales\footnote{\url{http://gov.wales/newsroom/educationandskills/2013/130114broadband/?lang=en}})
and to promote and support the use of digital resources and
technologies by learners and teachers.


This policy focus on the use of technology in education stimulated
focus on the academic disciplines of ICT and computer science in
Wales. In January 2013, the Welsh Government's Minister for Education
and Skills
announced\footnote{\url{http://gov.wales/about/cabinet/cabinetstatements/2013/ictsteeringgroup/?lang=en}}
the formation of an ICT Steering Group to consider the future of
computer science and ICT in schools in Wales, framed by the outcomes
of a Ministerial
announcement\footnote{\url{http://gov.wales/about/cabinet/cabinetstatements/2012/ictreview/?lang=en}}
and seminar in November 2012, attended by representatives from a range
of key stakeholders including schools, the National Digital Learning
Council, further education, higher education, awarding organisations,
industry and the media. There had been significant focus on computer
science education more broadly, as evidenced by a Ministerial speech
from June 2012:

% talk about CAS/Technocamps conference
\begin{quotation}
Computer science touches upon all three of my education priorities:
literacy, numeracy and bridging the gap. It equips learners with the
problem-solving skills so important in life and work.

The value of computational thinking, problem-solving skills and
information literacy is huge, across all subjects in the curriculum. I
therefore believe that every child should have the opportunity to
learn concepts and principles from computer science.

Indeed, computing is a high priority area for growth in Wales. The
future supply and demand for science, technology and mathematics
graduates is essential if Wales is to compete in the global economy.

It is therefore vitally important that every child in Wales has the
opportunity to study computer science.
\end{quotation}

Thus, the key themes derived from the seminar, as well as wider policy
developments, provided the following remit for the ICT Steering Group:

\begin{itemize}
\item ICT in schools needs to be re-branded, re-engineered and made
relevant to now and to the future;
\item Digital literacy is the start and not the end point -- learners
need to be taught to create as well as to consume;
\item Computer science should be introduced at primary school and
developed over the course of the curriculum so that learners can
progress into a career pathway in the sector.
\item Skills, such as creative problem-solving, should be reflected in
the curriculum; and,
\item Revised qualifications need to be developed in partnership with
schools, higher education and industry.
\end{itemize}

The membership of the ICT Steering Group was comprised of
representatives from a cross-section of key
stakeholders\footnote{N.B. The ICT Steering Group was co-chaired by
Crick, with Moller as a full member.}, tasked with providing clear
recommendations on the way forward for computer science and ICT in
Wales. It was initially envisaged to report back in autumn 2013, with
its recommendations informing the wider review of assessment and 14-19
qualifications\footnote{\url{http://gov.wales/topics/educationandskills/qualificationsinwales/revofqualen/?lang=en}},
with any necessary changes being considered as part of any revisions
to the National Curriculum in
Wales\footnote{\url{http://gov.wales/about/cabinet/cabinetstatements/2012/curriculumreview/?lang=en}}.

After eight months of discussion and evaluation, including an open
consultation and a number of stakeholder events through the summer of
2013, the ICT Steering Group published its
recommendations~\cite{welshictreview:2013} for a way forward for ICT
in Wales in October 2013, highlighting the importance of computing and
digital literacy in a modern, challenging and aspirational national
curriculum. Its headline recommendations were grouped into three main
themes: curriculum and qualifications, teacher training and
professional development, infrastructure and monitoring. Most
significantly, it recommended a new subject named Computing
(disaggregating into two main areas: Computer Science and Information
Technology) should be created to replace ICT from Foundation Phase
onwards, encouraging creativity, allow thematic working and develop
real world problem-solving This new subject should be integrated into
the curriculum as the fourth science, served by a mandatory Programme
of Study, and receive the same status as the other three sciences. It
recommended a clear distinction between the academic discipline of
computing and digital competencies by proposing a Statutory Digital
Literacy Framework to work alongside existing frameworks for literacy
and numeracy from Foundation Phase through to post-16 education. There
was also a strong focus on the ICT teaching professional in Wales,
particularly around initial teacher training and incentivising routes
into the profession, as well as raising the profile and importance of
career-long professional development.

% full recommendations:
% \begin{enumerate}
% \item A new subject named Computing should be created to replace
%   Information and Communications Technology (ICT) from Foundation
%   Phase onwards. This new subject will disaggregate into two main
%   areas: Computer Science (CS); and Information Technology (IT)
% \item Computing should be integrated into the curriculum as the fourth
%   science, served by a mandatory Programme of Study, and receive the
%   same status as the other three sciences.
% \item A Statutory Digital Literacy (DL) Framework should be
%   implemented to work alongside the Literacy and Numeracy Framework
%   from Foundation Phase through to post-16 education.
% \item Perceptions of Computing education pathways should be changed to
%   recognise the key societal roles of computing and technology, as
%   well as promote the importance and diversity of IT careers.
% \item The revised Computing curriculum should encourage creativity,
%   allow thematic working and develop real world problem-solving. It
%   should be flexible enough to continually evolve to remain current,
%   adopting an Agile ideology and approach to ensure this.
% \item A range of engaging and academically rigorous pathways and
%   bilingual qualifications for Computing and Digital Literacy should
%   be devised, encouraging interest and opportunities for deeper
%   learning.
% \item Engagement and collaboration between education and industry
%   should be an integral part of the curriculum to embed current
%   practices and skills.
% \item Pathways for Initial Teacher Training (ITT) in Computing should
%   be created to encourage the best talent into the profession. All
%   entrants to the teaching profession should have the skills to
%   deliver the Digital Literacy Framework (DLF).
% \item A programme of training and professional development to enable
%   the new Computing curriculum should be accessible to new and
%   existing teachers.
% \item A National Technology Framework should be devised to create an
%   effective technology infrastructure for education. Welsh Government,
%   local authorities, industry and learning providers should be
%   responsible for its effective implementation and strategic
%   development.
% \item Effective monitoring arrangements should be created for
%   Computing and the Digital Literacy Framework. Estyn should consider
%   relevant changes to the Common Inspection Framework in light of all
%   of these recommendations.
% \item An appropriate body or properly constituted group should oversee
%   the implementation of these recommendations. Its remit would need to
%   be broad enough to encompass this crucial governance role, utilising
%   appropriate expertise and representing key stakeholders.
% \end{enumerate}

In the context of the recently announced new computing curriculum in
England, the ICT Steering Group's report was well-received, addressing
the specificity of the educational challenges in Wales, as well as
providing an broad and balanced curriculum, from digital competencies
through to computer science. However, as this was an independent
review, the recommendations were non-binding and an official response
from the Welsh Government was only received in March
2014\footnote{\url{http://learning.gov.wales/docs/learningwales/publications/140324-response-to-the-ICT-steering-groups-report-en.pdf}}. In
this period, there had been further developments in the wider review
of the curriculum and assessment arrangements in Wales. While aspects
of the recommendations around digital competencies has been accepted,
anything directly relating to curriculum and qualifications had been
delegated due to announcement in March 2014 of a wholesale independent
review to provide recommendations to inform the development of a new
Curriculum for Wales. Thus the report of the ICT Steering Group formed
an important part of the evidence base for this review going forward.

Also to note: Issue of multitude of reviews, none of which recommend watering down
the subject; Linking to economic imperatives.

\subsection{2015 Donaldson Review}

%\textbf{Tom -- please write this~\cite{Donaldson:2015}.}

In March 2014 Professor Graham Donaldson, a former chief school
inspector in Scotland, was appointed by the Welsh Government to
conduct an independent fundamental review of curriculum and assessment
arrangements of the entire curriculum in Wales, from Foundation Phase
to Key Stage 4. This continued on from a number of previous national
consultations and reviews, such as the 2011-2012 Review of
Qualifications\footnote{\url{http://gov.wales/topics/educationandskills/qualificationsinwales/revofqualen/?lang=en}}
for 14 to 19-year-olds in Wales (which aimed to ensure that
qualifications in Wales are understood and valued and meet the needs of
young people and the Welsh economy).

TO DO: Key points to merge into text:

\begin{itemize}
\item structure of Foundation and Key Stages are going (moving away
  from the Ken Baker 1988 model!) -- unsurprisingly, feels quite
  similar to the Scottish Curriculum for Excellence model (link here);
\item replaced with six ``areas of learning'';
\item three cross-cutting "collective responsibilities": literacy, numeracy, digital competencies;
\item subjects should ``service the curriculum but not define it'';
\item lots of focus on creativity, as well as entrepreneurial activity;
\item big focus on citizenship, health and wellbeing (e.g. sex education, PSE);
\item assessment for learning, supporting excellence in learning and teaching;
\item Estyn need to change how they operate, promoting improvement not just testing'
\item Big hat-tip to the 2013 Review of the ICT curriculum, accepting
  importance of digital competencies, as well as computer science
  sitting within the new ``Science \& Technology'' area of learning.
\end{itemize}

\subsection{2015 Furlong Review and the New Deal}

\textbf{Tom -- please write this~\cite{Furlong:2015}.}

\begin{quote}\it
Key points in the Furlong review of Initial teacher
education~\cite{Furlong:2015} (but not huge amounts of computing
specific stuff here) and the New Deal (ref, again not huge amounts of
specificity to computing education, but refer to QTS standards with
regards to expectations around digital competencies). Also refer to
incentivisation of entrants to the teaching profession (linking back
to ICT review~\cite{welshictreview:2013}), comparing
Wales\footnote{\url{http://teachertrainingcymru.org/node/16}}
vs. England\footnote{\url{https://getintoteaching.education.gov.uk/bursaries-and-funding}
and \url{http://academy.bcs.org/content/eligibility}}.
\end{quote}

In March 2015, a ‘New Deal’ for the Education Workforce was proposed
by the Welsh Government, complementing the outcomes from the Donaldson
and Furlong reviews, to reshape continuing professional development
for teaching professionals to support them in shaping and delivering the new
curriculum. In this New Deal, career-long professional development is
a priority, with plans to introduce a new Professional Learning
Passport for teachers in Wales, as well as the Welsh Government
supporting schools to produce tailor made School Development Plans
which will have workforce development at their centre, engaging all
staff in high quality continuing professional development. More
specifically, it also proposed to revised the Professional Standards
for the education workforce that set out the professional skills and
knowledge required of practitioners to deliver a future curriculum and
embed initial qualification standards in a career long framework. 


TO DO: cite relevant OECD reports to provide international context?


\section{The Technocamps Effect}

Technocamps is a multi-faceted Universities-based
operation engaging with schools -- both their pupils and their teachers --
throughout Wales and across all ages. Its main activities are as follows.

\begin{description}
\item[Workshops]
One-day campus-based workshops offered to whole classes
to give the pupils an introduction to computing,
particularly computational thinking and problem solving.
The whole class approach allows us: to address the gender divide,
by engaging with an equal number of boys and girls;
and to engage with those with no predisposition (or indeed an aversion)
to digital technology, to create an interest of computing within them.
\item[Technoclubs]
Lunchtime clubs in schools where pupils develop
their computational thinking and building skills.
\item[Bootcamps]
Two-day campus-based workshops held during school holidays.
\item[After Schools Clubs]
Two-hour late afternoon sessions held on campus or in the community.
\item[Playground Computing]
Day-long in-school workshops which present
the fundamentals of computer science to primary school pupils
through playful activities which develop computational thinking
and problem solving skills, but do not involve computers.
\item[Technoteach]
Training sessions, typically in the form of 20-hour modules
delivered one evening per week over six weeks.
Technoteach also encompasses other standalone twilight sessions
as well as an annual teachers conference.
\item[NEET Engagement]
Week-long summer residential sessions run in partnership with
the municipal youth services in which young people identified
as NEET (Not in Employment, Education of Training)
carry out a variety of team-building exercises,
learn app development and compete to design and build the best app.
\item[Student Placements]
Computer Science students at the University are offered
the opportunity to gain university course credits through
placements -- one day per week -- as teaching assistants
in school computing/ICT classes.
\end{description}

All Technocamps activities are provided completely free of charge
for all of its participants. This represents a huge investment
on the part of the Universities, but Technocamps has also received
various sources of funding in support of its activities.
The main funders are as follows.
\begin{description}
\item[ESF] (October 2010 - September 2014) --
A four-year \pounds 6 million EU-funded project to engage with secondary schools across South West Wales and the Valleys. This project involved Technocamps hubs at Aberystwyth University, Bangor University and the University of South Wales Glamorgan. Some 9,000 pupils from more than 180 schools and colleges have benefited from this project, as well as their teachers.
\item[NESTA] (June 2013 - December 2014) --
An 18-month \pounds 46,000 project to support the Playground Computing programme. This funding allows for a teacher to be seconded for 18 months to Technocamps in order to go out to primary schools throughout South Wales every day to present workshops. It has seen some 5,000 pupils at over 50 primary schools enjoy multiple day-long visits.
\item[National Science Academy] (November 2013 - March 2015) --
A 17-month \pounds 24,000 project to support the Technoteach programme; this funding was mainly in support of teachers registering on our six-week Technoteach modules, specifically providing their schools an amount of teacher cover to facilitate their attendance on the module. Over 120 teachers have thus far benefited from this project.
\item[Welsh Government] (September 2014 - March 2016) --
An 18-month \pounds 370,000 project under the Welsh Government's Learning in Digital Wales (LiDW) Tender. The LiDW Tender is to deliver 3-hour taster sessions at each of the 210 state-sponsored secondary schools across Wales, and will be delivered by each of the six Technocamps hubs.
\end{description}

\section{Conclusions}
What are the top-line messages???
% data from ICT review survey?

% bib
\bibliographystyle{abbrv}
\bibliography{wipsce2015}

%\section*{To Do}
%
%In no particular order...
%
%\begin{itemize}
%\item Set UK context over past 3-5 years
%\item Link to English and Scottish changes
%\item Welsh context
%\item Technocamps: the ten year journey from 2003
%\item Convergence, the pan-Wales problem
%\item Key contributions: aims, targets, impacts, young people, NEETs, coverage
%\item Links with CAS Wales
%\item Hubs (both TC and CAS)
%\item Funding models: ESF, NSA, Nesta, etc
%\item {\textbf{Key theme}}: Building capacity, the problems of
%  England's NoE model in Wales
%\item UK policy: RS report, English curriculum, qualification change,
%  UKForCE, etc.
%\item Welsh policy: ICT curriculum, ICT review, Estyn reports, ICT
%  sector support/economic drivers, curriculum change,
%  Donaldson and post-Donaldson
%\item THE FUTURE...!
%\end{itemize}
%
%\subsection*{References to fit in}
%\begin{itemize}
%\item
%General CAS
%citations~\cite{crick+sentance:2011,brown-et-al-sigcse2012,brown-et-al-toce2014}.
%
%\item
%Teachers, CPD and
%NoE~\cite{sentance-et-al-wipsce2012,sentance-et-al:2013,sentance-et-al:2014}.
%
%\item
%Technocamps~\cite{ball-et-al:2012,boyle-et-al:2012}
%
%\item
%Welsh Government report:
%\begin{itemize}
%\item
%ICT Review~\cite{welshictreview:2013}
%\item
%Graham Report on STEM~\cite{STEMreview:2014}
%\item
%Donaldson Report~\cite{Donaldson:2015}
%\item
%Furlong Report~\cite{Furlong:2015}
%\end{itemize}
%
%\item
%Misc Reports
%\begin{itemize}
%\item
%NESTA Report~\cite{NESTA:2015}
%
%\end{itemize}
%\end{itemize}

\end{document}
