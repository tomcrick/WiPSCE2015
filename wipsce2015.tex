% This is ''sig-alternate.tex'' V2.0 May 2012
% This file should be compiled with V2.5 of '\'sig-alternate.cls'' May 2012
%
% This example file demonstrates the use of the \'sig-alternate.cls'
% V2.5 LaTeX2e document class file. It is for those submitting
% articles to ACM Conference Proceedings WHO DO NOT WISH TO
% STRICTLY ADHERE TO THE SIGS (PUBS-BOARD-ENDORSED) STYLE.
% The \'sig-alternate.cls' file will produce a similar-looking,
% albeit, 'tighter' paper resulting in, invariably, fewer pages.

\documentclass{sig-alternate}
\sloppy
\usepackage{paralist}
\usepackage{url}

\begin{document}
%
% --- Author Metadata here ---
\conferenceinfo{WIPSCE}{2015 London, UK}
\CopyrightYear{2015} % Allows default copyright year (20XX) to be over-ridden - IF NEED BE.
%\crdata{0-12345-67-8/90/01}  % Allows default copyright data (0-89791-88-6/97/05) to be over-ridden - IF NEED BE.
% --- End of Author Metadata ---

\title{Technocamps: \\[1ex] Advancing Computer Science Education in Wales}

%\title{Technocamps: A Decade of Supporting\\Computer Science Education in Wales}
% This title is so good I want to reserve it for TOCE/SIGCSE

\numberofauthors{2}
\author{
% 1st. author
\alignauthor
Tom Crick\\
\affaddr{Department of Computing}\\
\affaddr{Cardiff Metropolitan University, UK}\\
\affaddr{tcrick@cardiffmet.ac.uk}
% 2nd. author
\alignauthor
Faron Moller\\
\affaddr{Department of Computer Science}\\
\affaddr{Swansea University, UK}\\
\affaddr{f.g.moller@swansea.ac.uk}\\
}

\maketitle

\begin{abstract}
Computer science education in the UK over the past five years has
undergone significant scrutiny, upheaval and reform. From September
2014, we have seen the implementation and delivery of a new computing
curriculum in England, alongside long-term investment in the
professional development of teachers in Scotland. However, in Wales, a
devolved nation of the UK, numerous political, geographical and
socio-technical issues have hindered any substantive educational
policy or curriculum reform for computer science.

This is despite the widespread efforts to address the failings of
computer science education in schools since at least 2003 through
Technocamps, a pan-Wales university-based schools outreach
programme. In this paper we outline the history (and pre-history) of
Technocamps, contextualised by the devolved nature of education in the
UK, positioning Wales with its specific issues and
challenges. Furthermore, we present data both in support of this
university engagement and intervention model as well as its wider
positive effect on promoting and supporting computer science education
in Wales, a nation taking its first steps on the path of a large-scale
national curriculum review and significant educational reform.
\end{abstract}

% A category with the (minimum) three required fields
\category{K.3.2}{Computers \& Education}{Computer and Information Science Education}[Computer Science Education]
\category{K.4.1}{Computers And Society}{Public Policy Issues}
\keywords{Computer Science Education; High School; Teachers;
  Professional Development}

% TO DO: premable -- a note on terminology? e.g. ICT, etc?

\section{Introduction}\label{intro}
% adapted from 2013 TOCE pitch
In the early 1980s, the BBC Micro was introduced to schools throughout
the UK as part of the BBC's \emph{Computer Literacy Project}; before
long they were in 80\% of UK classrooms~\cite{vasko:1986}. By
encouraging young learners to experiment with computers, a generation
of creative (and computational) talent was spawned. Applications in
the UK to study computer science at university hit a peak, with
computer science graduates helping computers come to dominate every
aspect of our lives.

Fast forward 30 years and the situation is very much different. The
computer is no longer a novelty; children now typically spend more
time in front of a computer screen than a TV screen at home, but like
the TV, their interest is restricted to using the computer, not in
experimenting with it. Computer studies in school -- since the late
1990s generally called Information and Communications Technology (ICT)
-- has evolved into IT studies with an emphasis on digital literacy
and ``office productivity'' skills -- significantly more mundane than
the social networking and gaming for which many pupils use their home
digital devices. A full two-thirds of ICT teachers in the UK do not
have a relevant qualification but may have moved into the role of ICT
teacher simply by being sufficiently digitally
literate~\cite{RoyalSoc:2012}.  The situation is worse in Wales, where
this figure rises to 75\%~\cite{GTCW:2008}. Applications to study
computer science at university slumped in the early part of the
millennium -- especially amongst females -- and many of those who
started a university computer science degree course found themselves
dropping out during the first year, surprised at what computer science
is and what studying it entails.

In the early 2000s, the Department of Computer Science at Swansea
University started looking into ways to address this issue.
Unfortunately, attempts to reach out to teachers in local schools
faced great resistance, in part due to their lack of confidence in
teaching actual computer science as opposed to anything more
complicated than using desktop software packages.

As an alternative route into schools, Swansea University created
\emph{Technocamps} in 2003, an outreach programme which brings groups
of school children to the University campus for day-long workshops based on
selected computational themes to inform them what computing is about,
followed-up by support in setting up
extracurricular clubs -- \emph{Technoclubs} -- in the schools.
Technocamps proved very successful as a local initiative, with many
students studying computer science at Swansea University claiming to be
influenced by Technocamps activities.

In 2010, based on long-term empirical data regarding its effect on
school children's attitudes towards computing -- as well as their
teachers -- Swansea University was awarded \pounds 3.9 million funding
towards a \pounds 6 million four-year project (with the remaining
\pounds 2.1 million generated through matched funding from the
university) by the Welsh Government under the EU's European Social
Fund (ESF) Convergence
Programme\footnote{\url{http://wefo.wales.gov.uk/programmes/20072013/convergence/?lang=en}}
to run Technocamps as a pan-Wales project with regional hubs at the
Universities of Aberystwyth, Bangor and Glamorgan (now University of
South Wales)\footnote{Technocamps hubs have subsequently been set up
at most of the remaining major Universities in Wales, specifically
Cardiff University, Cardiff Metropolitan University, and Glynd\^wr
University in Wrexham.}  Though focusing on the children, Technocamps
also provides ``Technoteach'' events aimed at up-skilling ICT teachers
in Welsh schools.  Technocamps has since provided computing-related
activities and resources for tens of thousands of young people across
Wales, as well as interacting with hundreds of teachers across
hundreds of the nation's schools.

However, Technocamps is not alone in exploring solutions to the
multitude of problems in computer science education in Wales.  In
particular, in 2008 the Computing At School
(CAS)\footnote{\url{http://www.computingatschool.org.uk/}}
organisation was formed, which has since become recognised as the UK
subject association for computer science and a key stakeholder from a
policy perspective. Its current membership of over 18,000 teachers and
computing professionals work hard to promote the teaching of computer
science at school, predominantly in England.  However, whilst great
changes have taken place in England due in no small part to CAS
lobbying and on the ground initiatives -- underpinned by generous
funding of CAS by England's Department of Education -- the CAS effect
has been less noticeable in Wales, in part due to lack of central
funding, with the rapid curriculum changes pushed through in England
in many ways resisted by the Welsh Government.

Wales is a devolved nation within the UK, with its own elected
national government fully responsible for its education system.  At
the 2012 Annual Technocamps Teachers'
Conference\footnote{\url{http://www.technocamps.com/blog/boost-digital-literacy-and-computer-science}},
the Welsh Government's Minister for Education and Skills publicly
acknowledged the importance of computer science education for all --
noting the impact of Technocamps -- and expressed understanding of the
wider educational and socio-economic impact that the government can
make with educational reform in Wales.  However, with only 5\% of the
population of England and with its distinct geographical and
socio-cultural challenges, Wales presents a variety of unique
challenges in addressing curriculum reform. Nevertheless, since 2013
we have seen significant industry and public scrutiny of the relevancy
of the school curriculum and the changing skills demands of the wider
digital economy, with a range of government-initiated independent
reviews of ICT culminating ina substantial review of the wider
national curriculum. Wales is thus on the cusp of substantial reform,
with Technocamps and CAS having a frontline role in the development of
a new computing curriculum, as well as supporting the professional
development of teachers.

% check this is correct at the end!
In this paper, we explain the devolved nature of education in the UK
focusing on Wales with its specific challenges and issues
(Section~\ref{welshukedu}); describe the backdrop to Technocamps and
why it was created in the way it was (Section~\ref{compedu}); and
present data both in support of university intervention as well as the
positive effect this intervention is having (Section 4).  We finish
with a consideration of the challenges remaining in Welsh education
(Section 5).

% taken from TOCE paper, adapted for Welsh focus
\section{Wales and the UK's Education System}\label{welshukedu}

The UK originally consisted of four nations ruled by one parliament,
with an overall population of 64.1 million: England (population: 53.9
million), Scotland (5.3 million), Wales (3.1 million) and Northern
Ireland (1.8 million)~\cite{onspop:2014}. In 1997, Scotland and Wales
held referendums which determined in both cases the desire for
self-government.  In the case of Wales, this led to the Government of
Wales Act 1998 which created the National Assembly for Wales, to which
a variety of powers were devolved from the UK parliament on 1 July
1999 (and again with the Wales Act 2014).  In particular, education --
which until then was a UK-wide government portfolio (minus Scotland,
which for historical reasons has had a distinct legal and education
system from England and Wales) -- came under the control of the
National Assembly for Wales, under the direction of the Department for
Education and Skills (originally the Department for Children,
Education, Lifelong Learning and Skills).

% % change this to have a brief discussion about all of the main
% systems in UK, especially England and Scotland (imp: Curric. for
% Excellence, plus Computing Science), then focus on Wales...

Wales is a small nation to the west of England (see
Figure~\ref{fig:wales}(b)).  It has an ancient Celtic culture and a
thriving language (with c.20\% of the population able to speak Welsh).
Its south coast became pre-eminent during the Industrial Revolution
due to coal mining and heavy industry; however, Wales is mostly rural
and suffers from post-industrial poverty, seasonal employment and the
dependence on the public sector for a significant proportion of
jobs. The country is sparsely populated with resilience and
interconnectedness of the transport infrastructure.  Hence its
communities -- and more specifically its schools and teachers --
suffer from the perils of isolation. Apart from the south east corner
(including its capital Cardiff, with a population of c.325,000) and
the regions bordering England, the rest of the country is formally
designated by the EU as a so-called ``Convergence area'' (see
Figure~\ref{fig:wales}(a)), meaning its per-capita gross domestic
product (GDP) is less than 75\% of the EU average.

%\begin{figure}
%  \centering
%  \includegraphics[width=0.5\columnwidth]{images/wales.png}
%  \caption{Convergence region of Wales, encompassing 15 out of 22 local councils}
%  \label{fig:wales}
%\end{figure}

Prior to devolution, the education system in Wales was essentially
identical to that in England and was in a generally healthy state,
outperforming other regions in the UK in the years prior to and
immediately following devolution.  However, since devolution it has
suffered a rapid decline.  Evans carries out a systematic and detailed
analysis as to why this was the case, citing a multitude of policy
changes and poor interventions~\cite{Evans:2015}.

\begin{figure*}[htp]
\centering
\begin{picture}(445,250)(-5,-55)
%\put(-5,-55){\framebox(445,250){}}
\put(10,0){\includegraphics[width=0.31\textwidth]{images/wales.png}}
\put(275,0){\includegraphics[width=0.325\textwidth]{images/UK.png}}
\put(165,96){\line(6,-1){128}}
\put(10,1){\dashbox(155,190){}}
\put(293,45){\dashbox(58,68){}}
\put(90,-35){\makebox(0,0){\begin{tabular}[t]{@{\hspace{1em}}r@{ }l}
(a) &~Convergence area of Wales, which \\ & encompasses 15 (out
of 22) western \\ & councils that do \emph{not} border England
\end{tabular}}}
\put(360,-20){\makebox(0,0){(b)~Wales and England}}
\end{picture}
\caption{Maps of Wales and England}
\label{fig:wales}
\end{figure*}

Whilst broadly maintaining the general educational system of Key
Stages used in England (see Figure~\ref{fig:key-stages}), the Welsh
Government embarked on a 10-year revolutionary plan including:

\begin{figure}[!ht]
  \centering
  \includegraphics[width=0.9\columnwidth]{images/keystages.png}
  \caption{Key Stages in the English and Welsh education system}
  \label{fig:key-stages}
\end{figure}

\begin{itemize}
\item phasing out Standard Attainment Tests (SATs);
\item replacing the early Key Stage 1 with a learning through play-based Foundation Phase;
\item introducing the Welsh Baccalaureate at all levels: an
  overarching qualification, with a purely practical-based assessment
  mechanism, incorporating key skills; Wales, Europe and the world;
  work-related education; and personal and social education;
\item emphasising the focus on the Welsh language and  Welsh-medium
  schools;
\item addressing the abundance of small schools in the  predominantly
  rural communities throughout Wales;
\item tackling deprivation.
\end{itemize}

Much of this plan was lauded, being learner-focused, placing an
emphasis on skills development and ensuring that it is appropriate for
the specific needs of Wales. However, since its implementation, it has
been criticised for various reasons and by various stakeholders.  The
then Minister for Education and Skills appointed in June 2010, in
looking for the causes of Wales' failing education system, found cause
to commission no fewer than 24 costly reviews before his untimely
resignation in February 2013 -- almost one per
month~\cite{Evans:2015}.

With devolved government comes autonomy over fiscal matters; and the
correlation between money and performance is an obvious target for
critics, who point to a growing spending shortfall between Wales and
England.  The average spend per pupil in Wales in 2000-2001 -- just
after devolution -- was more than every region of England apart from
the large metropolitan areas of London, the West Midlands and the
North West, all of which benefit from their vast economies of scale.
However, since then, the gap between the education budgets per pupil
between Wales and England has steadily grown by about 1\% per year;
the figures forecast for 2013-2014 show 13\% more being spent per
pupil in England than in Wales (\pounds7,533 per pupil in England as
opposed to \pounds6,676 per pupil in Wales)~\cite{Evans:2015}.

\section{Computing Education in the UK}\label{compedu}

In the 1980s, computer studies was a popular subject in schools across
the UK. As mentioned previously, the ubiquitous presence, in both
schools and homes, of the popular BBC Micro -- which was useful for
little else unless you were able to program -- saw a large proportion
of school children learning the fundamentals of programming in a
curriculum which included a variety of complementary topics such as
hardware, software, Boolean logic and binary number
representation~\cite{Doyle:1988}.

By the 1990s, however, the emergence of pre-installed software
packages -- specifically office productivity software such as word
processors and spreadsheet programs -- meant that computers were no
longer predominantly machines that needed to be programmed in order to
do anything useful or interesting.  Less and less time was being spent
in the computer studies classroom on thinking about and writing
programs, as basic digital literacies and IT skills became regarded as
the priority. However, as interest in viewing the computer as a
creative tool waned in favour of using it for more mundane tasks,
various problems were being created, which were highlighted in two
independent enquiries in 1997: the McKinsey
Report~\cite{McKinsey:1997} and the Stevenson
Report~\cite{Stevenson:1997}.  Both reports concluded that Information
Technology in UK schools was in a primitive state and in need of
attention and major investment. In line with the Stevenson Report,
computer studies evolved into a new subject whose name was coined in
that same report: {\emph{Information and Communications Technology}}
(ICT).  Over the decade starting in 1997, the UK Government invested
over \pounds3.5 billion in ICT in schools through various initiatives
such as the National Grid for Learning (NGfL) and the New
Opportunities Fund (NOF)~\cite{Doughty:2006}.

By 2000, then, ICT had permeated both primary and secondary school
curricula, especially in the newly-devolved nations. The emphasis was
on developing the children's IT skills and digital literacy in an
honest attempt to address the increasing need for digital
compentencies amongst the general public.  However, despite enormous
government-funded ICT initiatives, various reports throughout the
decade identified problems with implementing government policy on ICT
educational
reform~\cite{OpieFukuyo:2000,Ofsted:2001,Ofsted:2002,Ofsted:2004,
Loveless:2005}. Younie summarises the problems identified by these
reports into five key areas, three being management and the other two
being: teacher training and competence; and impact on
pedagogy~\cite{Younie:2006}. The ICT curriculum in
Wales~\cite{welshictcurric:2008}, while generally viewed to be more
flexible and less prescriptive than the equivalent subject in England,
exhibited many of the same issues~\cite{estynict:2014}.

A decade later, a report by the Royal Society~\cite{RoyalSoc:2012},
the UK's premier science academy, made the very same observations.
The report noted that ICT suffers from a poor reputation amongst
pupils, parents and industry, who consider it dull and unchallenging
and hence a low-value discipline, especially compared to other
strategically-significant STEM subjects.  With ICT embedded across the
primary school curriculum, secondary school pupils find ICT in
secondary school neither stimulating nor engaging. The Wolf
Report~\cite{Wolf:2011} further notes that the undemanding nature of
ICT qualifications in secondary schools is readily exploited by the
schools: due to a high league table weighting associated with
vocational qualifications, easily-achieved high results in ICT offer a
welcome boost to a school's league table position. Furthermore, as ICT
is typically presented by schools as their ``computing'' offering,
students who might otherwise enjoy studying computer science are
actively put-off from what they are led to believe is computer
science.

\section{Technocamps}\label{technocamps}

\subsection{Technocamp and Swansea University}

As experienced by other UK universities, the numbers of students
enrolling in computer science degree programmes at Swansea University
increased through the end of the millennium due to the dot-com boom.
However, as depicted in Figure~\ref{fig:numbers}, throughout the UK
(as elsewhere) the numbers then peaked, and what followed was a steady
five-year decline, dropping more than 40\% during that period, with
the worst effect on the already-dwindling numbers of female students.
Even at its peak, more than a third of students who started a computer
science degree programme left the programme before their second year
of study, citing a mistaken understanding of the subject as their
primary reason for leaving.

\begin{figure}[!ht]
  \centering
  \includegraphics[width=0.9\columnwidth]{images/numbers.png}
  \caption{Applications to University computer science degree programmes
           in the UK (top) and Wales (bottom), males (blue) and females (pink).
           (source: Universities and Colleges Admissions Service
           (UCAS))}
  \label{fig:numbers}
\end{figure}

In an attempt to address this worrying anomaly, the Department of
Computer Science at
Swansea\footnote{\url{http://www.swansea.ac.uk/compsci/}} reached out
to local secondary school ICT teachers, inviting them to meetings at
the university, and offering to visit schools to discuss the subject
with the teachers and to give motivational talks to students. Indeed,
the Department was invited every year to a number of schools in
England to present such talks to school children making their
university admissions selections.  However, interest locally was more
than absent: there was positive resistance to the department giving
talks to their prospective university applicants; such activity was
typically characterised as merely nakedly ``pitching for students.''
In reality, for reasons explained later which did not apply to
teachers in England, teachers in Wales were generally feeling
over-burdened and disinterested in exploring any perceptions of
inadequacy in the curriculum and their delivery.

As it appeared to be futile to influence schools and their ICT
teachers directly, Technocamps was created in 2003 to promote
computing amongst their pupils.  This was a programme of engaging
interactive computational workshops taking place on the university
campus whose ultimate aim was to subtly re-introduce computer science
into the ICT curriculum by generating the demand from the students.
Originally run only at Swansea University, Technocamps hubs have since
been created at most universities throughout Wales, offering wide
geographical coverage.

Teachers in Wales were happy to ``treat'' their classes to these ``day
out'' activities; but they were then faced with the prospect of
satisfying their pupils' newly-discovered passion for computing,
programming and computational thinking by introducing ``Technoclubs''
as lunch-time extra-curricular activities in the school.  With
generous help, resources and guidance from Technocamps -- along with
the fact that in many cases students appeared to be more technically
informed and digitally literate than their teachers -- these clubs
have flourished, and the impact of Technocamps in changing attitudes
in Welsh schools regarding ICT and computing has been widely
acknowledged, both by the Welsh Government and National Assembly for
Wales, as well as the teaching community in Wales.  An independent
review of Technocamps activity in the (socio-economically
disadvantaged) Convergence area of Wales carried out for Welsh
Government estimates that 5\% of Welsh secondary students (ie, aged
11-19) have engaged with Technocamps through Workshops, and that more
than a quarter of the secondary schools in the region have established
Technoclubs~\cite{Wavehill:2015}.

\subsection{The Technocamps Effect}

As can be seen, Wales provides a variety of major challenges
-- political, geographical, geo-political, social --
in changing its curriculum to re-introduce Computer Science
into the space populated by ICT.
There is relatively little interest in this
amongst schools, teachers and politicians;
and any attempt at creating change would require
vast external energy and resources.

\textbf{Technocamps} was created to take up this challenge.
It is a multi-faceted Universities-based
operation engaging with schools -- both their pupils and their teachers --
throughout Wales and across all ages. Its main activities are as follows.

\begin{description}
\item[Workshops]
One-day campus-based workshops offered to whole classes
to give the pupils an introduction to computing,
particularly computational thinking and problem solving.
The whole class approach allows us: to address the gender divide,
by engaging with an equal number of boys and girls;
and to engage with those with no predisposition (or indeed an aversion)
to digital technology, to create an interest of computing within them.
\item[Technoclubs]
Lunchtime clubs in schools where pupils develop
their computational thinking and building skills.
\item[Bootcamps]
Two-day campus-based workshops held during school holidays.
\item[After Schools Clubs]
Two-hour late afternoon sessions held on campus or in the community.
We run two types of such clubs: one standard computing club
in which participants get lessons, tutoring and individual help
on all manner of programming tasks, be they Python or Visual Basic
or HTML/CSS or RobotC or an intricate Arduino robotics project;
and the other on computational thinking, called the Logic Club,
in which the participants work on problem-solving tasks,
typically developing step-by-step algorithmic solutions
to a series of problems of varying difficulty.
\item[Playground Computing]
Day-long in-school workshops which present
the fundamentals of computer science to primary school pupils
through playful activities which develop computational thinking
and problem solving skills, but do not involve computers.
\item[Technoteach]
Training sessions, typically in the form of 20-hour modules
delivered one evening per week over six weeks.
The Technoteach modules have been accredited by ASFI
-- Accredited Skills For Industry --
for their Certificate in Computing for Teaching.
Technoteach also encompasses other standalone twilight sessions
as well as an annual teachers conference.
\item[NEET Engagement]
Week-long summer residential sessions run in partnership with
the municipal youth services in which young people identified
as NEET (Not in Employment, Education of Training)
carry out a variety of team-building exercises,
learn app development and compete to design and build the best app.
\item[Student Placements]
Computer Science students at the University are offered
the opportunity to gain university course credits through
placements -- one day per week -- as teaching assistants
in school computing/ICT classes.
\end{description}

All Technocamps activities are provided completely free of charge
for all of its participants. This represents a huge investment
on the part of the Universities, but Technocamps has also received
various sources of funding in support of its activities.
The main funders are as follows.
\begin{description}
\item[ESF] (October 2010 - September 2014) --
A four-year \pounds 6 million EU-funded project to engage with secondary schools across South West Wales and the Valleys. This project involved Technocamps hubs at Aberystwyth University, Bangor University and the University of South Wales Glamorgan. Some 9,000 pupils from more than 180 schools and colleges have benefited from this project, as well as their teachers.
\item[NESTA] (June 2013 - December 2014) --
An 18-month \pounds 46,000 project to support the Playground Computing programme. This funding allows for a teacher to be seconded for 18 months to Technocamps in order to go out to primary schools throughout South Wales every day to present workshops. It has seen some 5,000 pupils at over 50 primary schools enjoy multiple day-long visits.
\item[National Science Academy] (November 2013 - March 2015) --
A 17-month \pounds 24,000 project to support the Technoteach programme; this funding was mainly in support of teachers registering on our six-week Technoteach modules, specifically providing their schools an amount of teacher cover to facilitate their attendance on the module. Over 120 teachers have thus far benefited from this project.
\item[Welsh Government] (September 2014 - March 2016) --
An 18-month \pounds 370,000 project under the Welsh Government's Learning in Digital Wales (LiDW) Tender. The LiDW Tender is to deliver 3-hour taster sessions at each of the 210 state-sponsored secondary schools across Wales, and will be delivered by each of the six Technocamps hubs.
\item[National Science Academy] (April 2015 - March 2016) --
A three-year \pounds 120,000 grant to support our Technoteach and
Playground Computing programmes.
\end{description}

\subsection{Government Recognition}

The impact described above that
Technocamps has had on changing perceptions in schools
has translated into impact on Welsh (and UK) Government thinking
within different legislative Departments.
We can cite a variety of points which evidence this fact.

\begin{itemize}

\item
In his Keynote Speech at the 2012 Annual Technocamps Teachers' Conference,
the Welsh Government's Minister for Education and Skills
publicly acknowledged the importance of computer science education;
noted that he is a key supporter of Technocamps;
and expressed understanding of the wider educational and
socio-economic impact that the government can make with reform in Wales.
He took the opportunity that this Keynote Speech afforded him
to announce a variety of funded initiatives to
support Technocamps' aims of embedding computing within
the school curriculum at all levels.

\item
One of the initiatives the Minister announced in his Speech
was the creation of a government panel
-- the National Digital Learning Council (NDLC) --
which would work on scoping the route forward for
his Department; and in his Speech he appointed
the Director of Technocamps as an Expert Advisor
to this panel.

\item
The Minister also commissioned a Review of the ICT Curriculum,
and appointed the Director of Technocamps onto the Review Panel.

\item
The Director of Technocamps was also appointed
to the Government's
Cross Party Group on Science and Technology.

\item
Technocamps has been recognised by the UK Government
as the driving force for computing education in Wales,
through an invitation to appear at the Houses of Parliament
on 29~October 2014,
hosted by the Chair of the Science and Technology Select Committee.

\item
The impact that Technocamps has had on schools
in the convergence area of Wales has been recognised
by the \emph{Department of Education and Skills} which has contracted
Technocamps to deliver workshops at \textbf{every}
state-sponsored secondary school throughout
the whole country between September 2014
and March 2016 as part of their
\emph{Learning in Digital Wales} programme.

\item
The impact that Technocamps has had on teachers has been
recognised by the \emph{Department of Economy, Science and Transport},
through the National Science Academy,
which has contracted Technocamps to deliver its 20-hour Technoteach module
between April 2015 and March 2018.

\item
The impact that Technocamps has had on primary schools has also been
recognised by the \emph{Department of Economy, Science and Transport},
again through the National Science Academy,
which has contracted Technocamps to deliver its Playground Computing
programme between April 2015 and March 2018.

\end{itemize}

%As reported by Hubweiser et al.~\cite{hubwieser-et-al:2011}, when establishing
%a model for viewing school CS education, it is apparent that there is
%much diversity between school education systems, and this can create
%an obstacle when trying to understand progress made in a different
%country. Here we describe the context of school education in the UK.

% also: quals reform -- talk about WJEC and Wales in context of GCSEs
% and A-Levels!

%compulsory schooling until age 16.
%All subjects are
%compulsory until the end of Key Stage 3 (KS3) and then students can
%choose approximately ten subjects to study for the next two years,
%which each lead to GCSE (General Certificate of Secondary Education)
%qualifications. However, while the National Curriculum in England and
%Wales are broadly similar, they are distinct and use different
%terminology.

% adjust to be Wales-focused
%There is state provision for education in the UK up to the age of 19,
%with mostly comprehensive, mixed ability schools across the UK. A few
%areas in England have retained a system of selective 11+ schools
%called grammar schools, which require students to sit an exam prior to
%entry, but these schools are in the minority. As well as state
%schools, 10\% of schools in the UK are independent fee-paying
%schools. Overall, in England there are approximately 24,000 schools,
%including 16,800 primary schools, 3,400 secondary schools and 2,400
%independent schools (primary and secondary).  However, the primary and
%independent schools tend to be smaller: the state-funded schools had
%4.2 million primary pupils and 3.2 million secondary pupils, with 0.6
%million pupils in independent schools.

%The ICT curriculum in Wales
%(2008)\footnote{\url{http://wales.gov.uk/topics/educationandskills/schoolshome/curriculuminwales/arevisedcurriculumforwales/nationalcurriculum/ictnc/?lang=en}},
%was perceived to be less prescriptive than the ICT curriculum in
%England, but exhibiting many of the same issues. It was recently
%reviewed by an independent steering group appointed by the Welsh
%Government~\cite{welshictreview:2013}, making clear recommendations for
%reforming the ICT curriculum as part of a broader national curriculum
%review for September 2014.
%
% update end of this section so that it leads into the Donaldson review...

\section{Recent Policy Developments}

\subsection{2013 ICT Curriculum Review}

% \textbf{Tom -- this is your story~\cite{welshictreview:2013};
% Please write it.}

In light of significant reform since devolution of education policy to
Wales in 1999, there has been much focus on the use of technology in
education. In September 2011, the Minister for Education and Skills
commissioned a review of ``digital classroom teaching'', setting up an
independent group to identify which digital classroom delivery aspects
should be adopted to transform learning and teaching for those aged 3
to 19. In particularl this review focused on how e-infrastructural
issues, such as high-quality, accessible digital classroom content
could be developed, but specifically on how teachers might get the
digital teaching skills to use ICT to transform schools. Their
report~\cite{haywarddigwales:2012}, published in March 2012,
highlighted how digital technologies, combined with sound pedagogy,
can make a substantial difference to learning experiences and
performance, recommends actions to introduce, embed and promote the
use of digital technologies in
education\footnote{\url{http://gov.wales/topics/educationandskills/publications/wagreviews/digital/?lang=en}}. Alongside
the commitment of significant funding for e-infrastructure to support
learning and teaching in Wales, in September 2012 the Welsh Government
established the National Digital Learning Council\footnote{N.B. Crick
is a member of the NDLC (2012-present), with Moller a special
advisor.}\footnote{\url{http://gov.wales/about/cabinet/cabinetstatements/2012/learningindigitalwales/?lang=en}}
to provide expert and strategic guidance on the use of digital
technology in teaching and learning in Wales.  The remit of the
Council was to guide the implementation of the {\emph{Learning in
Digital Wales}} programme (a strategic investment on next-generation
connectivity for schools in
Wales\footnote{\url{http://gov.wales/newsroom/educationandskills/2013/130114broadband/?lang=en}})
and to promote and support the use of digital resources and
technologies by learners and teachers.


This policy focus on the use of technology in education stimulated
focus on the academic disciplines of ICT and computer science in
Wales. In January 2013, the Welsh Government's Minister for Education
and Skills
announced\footnote{\url{http://gov.wales/about/cabinet/cabinetstatements/2013/ictsteeringgroup/?lang=en}}
the formation of an ICT Steering Group to consider the future of
computer science and ICT in schools in Wales, framed by the outcomes
of a Ministerial
announcement\footnote{\url{http://gov.wales/about/cabinet/cabinetstatements/2012/ictreview/?lang=en}}
and seminar in November 2012, attended by representatives from a range
of key stakeholders including schools, the National Digital Learning
Council, further education, higher education, awarding organisations,
industry and the media. There had been significant focus on computer
science education more broadly, as evidenced by a Ministerial speech
from June 2012:

% talk about CAS/Technocamps conference
\begin{quotation}
Computer science touches upon all three of my education priorities:
literacy, numeracy and bridging the gap. It equips learners with the
problem-solving skills so important in life and work.

The value of computational thinking, problem-solving skills and
information literacy is huge, across all subjects in the curriculum. I
therefore believe that every child should have the opportunity to
learn concepts and principles from computer science.

Indeed, computing is a high priority area for growth in Wales. The
future supply and demand for science, technology and mathematics
graduates is essential if Wales is to compete in the global economy.

It is therefore vitally important that every child in Wales has the
opportunity to study computer science.
\end{quotation}

Thus, the key themes derived from the seminar, as well as wider policy
developments, provided the following remit for the ICT Steering Group:

\begin{itemize}
\item ICT in schools needs to be re-branded, re-engineered and made
relevant to now and to the future;
\item Digital literacy is the start and not the end point -- learners
need to be taught to create as well as to consume;
\item Computer science should be introduced at primary school and
developed over the course of the curriculum so that learners can
progress into a career pathway in the sector.
\item Skills, such as creative problem-solving, should be reflected in
the curriculum; and,
\item Revised qualifications need to be developed in partnership with
schools, higher education and industry.
\end{itemize}

The membership of the ICT Steering Group was comprised of
representatives from a cross-section of key
stakeholders\footnote{N.B. The ICT Steering Group was co-chaired by
Crick, with Moller as a full member.}, tasked with providing clear
recommendations on the way forward for computer science and ICT in
Wales. It was initially envisaged to report back in autumn 2013, with
its recommendations informing the wider review of assessment and 14-19
qualifications\footnote{\url{http://gov.wales/topics/educationandskills/qualificationsinwales/revofqualen/?lang=en}},
with any necessary changes being considered as part of any revisions
to the National Curriculum in
Wales\footnote{\url{http://gov.wales/about/cabinet/cabinetstatements/2012/curriculumreview/?lang=en}}.

After eight months of discussion and evaluation, including an open
consultation and a number of stakeholder events through the summer of
2013, the ICT Steering Group published its
recommendations~\cite{welshictreview:2013} for a way forward for ICT
in Wales in October 2013, highlighting the importance of computing and
digital literacy in a modern, challenging and aspirational national
curriculum. Its headline recommendations were grouped into three main
themes: curriculum and qualifications, teacher training and
professional development, infrastructure and monitoring. Most
significantly, it recommended a new subject named Computing
(disaggregating into two main areas: Computer Science and Information
Technology) should be created to replace ICT from Foundation Phase
onwards, encouraging creativity, allow thematic working and develop
real world problem-solving This new subject should be integrated into
the curriculum as the fourth science, served by a mandatory Programme
of Study, and receive the same status as the other three sciences. It
recommended a clear distinction between the academic discipline of
computing and digital competencies by proposing a Statutory Digital
Literacy Framework to work alongside existing frameworks for literacy
and numeracy from Foundation Phase through to post-16 education. There
was also a strong focus on the ICT teaching professional in Wales,
particularly around initial teacher training and incentivising routes
into the profession, as well as raising the profile and importance of
career-long professional development.

% full recommendations:
% \begin{enumerate}
% \item A new subject named Computing should be created to replace
%   Information and Communications Technology (ICT) from Foundation
%   Phase onwards. This new subject will disaggregate into two main
%   areas: Computer Science (CS); and Information Technology (IT)
% \item Computing should be integrated into the curriculum as the fourth
%   science, served by a mandatory Programme of Study, and receive the
%   same status as the other three sciences.
% \item A Statutory Digital Literacy (DL) Framework should be
%   implemented to work alongside the Literacy and Numeracy Framework
%   from Foundation Phase through to post-16 education.
% \item Perceptions of Computing education pathways should be changed to
%   recognise the key societal roles of computing and technology, as
%   well as promote the importance and diversity of IT careers.
% \item The revised Computing curriculum should encourage creativity,
%   allow thematic working and develop real world problem-solving. It
%   should be flexible enough to continually evolve to remain current,
%   adopting an Agile ideology and approach to ensure this.
% \item A range of engaging and academically rigorous pathways and
%   bilingual qualifications for Computing and Digital Literacy should
%   be devised, encouraging interest and opportunities for deeper
%   learning.
% \item Engagement and collaboration between education and industry
%   should be an integral part of the curriculum to embed current
%   practices and skills.
% \item Pathways for Initial Teacher Training (ITT) in Computing should
%   be created to encourage the best talent into the profession. All
%   entrants to the teaching profession should have the skills to
%   deliver the Digital Literacy Framework (DLF).
% \item A programme of training and professional development to enable
%   the new Computing curriculum should be accessible to new and
%   existing teachers.
% \item A National Technology Framework should be devised to create an
%   effective technology infrastructure for education. Welsh Government,
%   local authorities, industry and learning providers should be
%   responsible for its effective implementation and strategic
%   development.
% \item Effective monitoring arrangements should be created for
%   Computing and the Digital Literacy Framework. Estyn should consider
%   relevant changes to the Common Inspection Framework in light of all
%   of these recommendations.
% \item An appropriate body or properly constituted group should oversee
%   the implementation of these recommendations. Its remit would need to
%   be broad enough to encompass this crucial governance role, utilising
%   appropriate expertise and representing key stakeholders.
% \end{enumerate}

In the context of the recently announced new computing curriculum in
England, the ICT Steering Group's report was well-received, addressing
the specificity of the educational challenges in Wales, as well as
providing an broad and balanced curriculum, from digital competencies
through to computer science. However, as this was an independent
review, the recommendations were non-binding and an official response
from the Welsh Government was only received in March
2014\footnote{\url{http://learning.gov.wales/docs/learningwales/publications/140324-response-to-the-ICT-steering-groups-report-en.pdf}}. In
this period, there had been further developments in the wider review
of the curriculum and assessment arrangements in Wales. While aspects
of the recommendations around digital competencies has been accepted,
anything directly relating to curriculum and qualifications had been
delegated due to announcement in March 2014 of a wholesale independent
review to provide recommendations to inform the development of a new
Curriculum for Wales. Thus the report of the ICT Steering Group formed
an important part of the evidence base for this review going forward.

TO DO/also to note: Issue of multitude of reviews~\cite{Evans:2015}, none of which recommend watering down
the subject; Linking to economic imperatives/priority sectors, brief
narrative at start?

\subsection{2015 Donaldson Review}

In March 2014 Professor Graham Donaldson, a former chief school
inspector in Scotland, was appointed by the Welsh Government to
conduct an independent fundamental review of curriculum and assessment
arrangements of the entire curriculum in Wales, from Foundation Phase
to Key Stage 4. This continued on from a number of previous national
consultations and reviews, such as the 2011-2012 Review of
Qualifications\footnote{\url{http://gov.wales/topics/educationandskills/qualificationsinwales/revofqualen/?lang=en}}
for 14 to 19-year-olds in Wales (which aimed to ensure that
qualifications in Wales are understood and valued and meet the needs
of young people and the Welsh economy), as well as aggregating a
number of independent subject-specific reviews, for example arts,
physical education and {\emph{Y Cwricwlwm Cymreig}} (the Welsh
cultural dimension of the curriculum), along with the 2013 ICT review.

The publication of the Donaldson report (``Successful
Futures'')~\cite{Donaldson:2015} in February 2015 proposed profound
changes to the education system in Wales. While identifying strengths
in the current education system, for example the Foundation Phase and
the commitment to the Welsh language and culture, the report
identified significant shortcomings of the current curriculum
arrangements, which essentially remain as devised in 1988 (when it
shared a national curriculum with England). The report argued that the
curriculum has become overloaded, complicated and, in many parts,
outdated. The report identifies four purposes for the curriculum,
recommending that the entirety of the school curriculum should be
designed to help all children and young people to become: ambitious,
capable learners, ready to learn throughout their lives; enterprising,
creative contributors, ready to play a full part in life and work;
ethical, informed citizens of Wales and the world; and healthy,
confident individuals, ready to lead fulfilling lives as valued
members of society.

More specifically for computing education and the role of technology,
the review identifies three cross-cutting, whole-schools ``collective
responsibilities'': literacy, numeracy and digital competencies. With
the structure of Foundation and Key Stages disappearing, individual
curriculum subjects would be replaced with six ``areas of learning'',
in which subjects should ``service the curriculum but not define
it''. All teaching and learning would be directed to achieving the
four curriculum purposes.

The Donaldson review recognised and adopted many of the
recommendations of the 2013 ICT review, recognising the importance
(and separating) digital compentencies from the curriculum subject of
computing, but providing clear pathways as well as significant
opportunities for cross-curricular learning across science and
mathematics. Computer science would thus sit within a new Science \&
Technology area, with a clear strand of learning from aged five
through to qualifications at 16 and 18. Furthermore, it recommended a
programme of professional learning to be developed to ensure that the
implications of the review for the skills and knowledge of teachers
are fully met, although no timescale for delivery were proposed (due
to legislative changes required). This curriculum review was
cautiously well-received by the education community and the
media\footnote{\url{http://www.bbc.co.uk/news/uk-wales-31534284}} in
Wales, with significant detail remaining to be seen in implementation,
resourcing and timescales.

% TO DO: Key points to merge into text:

% \begin{itemize}
% \item structure of Foundation and Key Stages are going (moving away
%   from the Ken Baker 1988 model!) -- unsurprisingly, feels quite
%   similar to the Scottish Curriculum for Excellence model (link here);
% \item replaced with six ``areas of learning'';
% \item three cross-cutting "collective responsibilities": literacy, numeracy, digital competencies;
% \item subjects should ``service the curriculum but not define it'';
% \item lots of focus on creativity, as well as entrepreneurial activity;
% \item big focus on citizenship, health and wellbeing (e.g. sex education, PSE);
% \item assessment for learning, supporting excellence in learning and teaching;
% \item Estyn need to change how they operate, promoting improvement not just testing'
% \item Big hat-tip to the 2013 Review of the ICT curriculum, accepting
%   importance of digital competencies, as well as computer science
%   sitting within the new ``Science \& Technology'' area of learning.
% \end{itemize}

\subsection{2015 Furlong Review and the New Deal}

The publication of the Donaldson curriculum review was quickly
followed by a review of initial teacher education in March 2015, led
by Professor John Furlong. His review~\cite{Furlong:2015} identified
that teacher training was at a ``critical turning point'' and needed
to be changed, with a vacuum in the leadership in Wales, substantial
underinvestment and support for staffing by universities in their
education departments and faculties, falling well short of best
practice in other parts of the UK and internationally. While not
specifically addressing issues for individual subjects, it notes the
raise in expectations of the proposed curriculum from the Donaldson
review (both in subject knowledge, as well as delivery), as well as
explicitly referring to expectations of digital competencies in the
qualified teacher standards, as well as how best to incentivise the
best applicants to enter the teaching profession in Wales
(e.g. current situation for funding in
Wales\footnote{\url{http://teachertrainingcymru.org/node/16}}
vs. England\footnote{\url{https://getintoteaching.education.gov.uk/bursaries-and-funding}
and \url{http://academy.bcs.org/content/eligibility}}), again linking
to the recommendations of the 2013 ICT review.

% \begin{quote}\it
% Key points in the Furlong review of Initial teacher
% education~\cite{Furlong:2015} (but not huge amounts of computing
% specific stuff here) and the New Deal (ref, again not huge amounts of
% specificity to computing education, but refer to QTS standards with
% regards to expectations around digital competencies). Also refer to
% incentivisation of entrants to the teaching profession (linking back
% to ICT review~\cite{welshictreview:2013}),.
% \end{quote}

The Furlong review was announced alongside the Welsh Government's ``New
Deal'' for the Education Workforce, complementing the outcomes from the
previous reviews, to reshape continuing professional development for
teaching professionals to support them in shaping and delivering the
new curriculum. In this New Deal, career-long professional development
is a priority, with plans to introduce a new Professional Learning
Passport for teachers in Wales, as well as the Welsh Government
supporting schools to produce tailor made School Development Plans
which will have workforce development at their centre, engaging all
staff in high quality continuing professional development. More
specifically, it also proposed to revised the Professional Standards
for the education workforce that set out the professional skills and
knowledge required of practitioners to deliver a future curriculum and
embed initial qualification standards in a career long framework.


TO DO: cite relevant OECD reports to provide international context?



\section{Challenges: Wales Divided}

Due to its EU funding restrictions during 2011-2014,
Technocamps was prohibited from providing any resourced support
(specifically, manpower for Workshops, teacher sessions,
Technoclub support, etc)
to schools outside of the convergence area -- namely,
the eastern region of Wales, including its capital city Cardiff,
bordering England
(see Figure~\ref{fig:wales}(a)).
%\begin{figure*}
%\centering
%\begin{picture}(445,250)(-5,-55)
%%\put(-5,-55){\framebox(445,250){}}
%\put(10,0){\includegraphics[width=0.33\textwidth]{images/wales.png}}
%\put(275,0){\includegraphics[width=0.35\textwidth]{images/UK.png}}
%\put(165,96){\line(6,-1){128}}
%\put(10,1){\dashbox(155,190){}}
%\put(293,45){\dashbox(58,68){}}
%\put(90,-35){\makebox(0,0){\begin{tabular}[t]{@{\hspace{1em}}l}
%Convergence Area of Wales, which \\ encompasses 15 (out
%of 22) western \\ councils that do \emph{not} border England
%\end{tabular}}}
%\put(360,-20){\makebox(0,0){Wales and England}}
%\end{picture}
%\caption{Maps of Wales and England}
%\label{fig:wales}
%\end{figure*}
Whilst truly unfortunate,
a fortuitous side effect of this restriction was that it allows for
a true assessment of the real impact of Technocamps,
as the country was invariably divided into two halves:
West Wales received the full Technocamps experience,
whilst East Wales (including its capital Cardiff) did not.

Cardiff is the base of the Welsh branch of Computing At School (CAS),
a teacher-led organisation in England which has successfully
lobbied for curriculum change in England; and
Technocamps supported the Chair of CAS Wales
in promoting
the teacher-led initiatives promoted by CAS.
In particular, in 2010 we jointly sent out an information pack
to every secondary school in Wales.
Technocamps produced all of the packs and
posted these out to all of the schools; CAS Wales provided
the postal costs for sending the information packs to the schools
outside of the convergence area of Wales.
(CAS Wales has a budget of \pounds100,000 from Welsh Government
to advance the CAS model of teacher-led activity across Wales,
supplementing the several millions of pounds granted
to CAS by the English Government for this activity across England.)
The information pack included full details of the extensive resources
being supplied on the Technocamps and CAS websites,
which schools and teachers could freely download and use,
in particular in support of extra-curricular computing clubs.

Despite the financial support CAS Wales receives, and the support it
offers teachers in Wales, the CAS model has never proven successful
in Wales. It is noticeable, for example, that whilst the Heads of
CAS England, CAS Scotland and CAS Northern Ireland are naturally all
teachers, only the head of CAS Wales, Dr Tom Crick, stands out:
he is not a teacher, but rather a University academic.
Equally, whilst CAS hubs across the UK are run by schools
for schools, abiding to the principle of the teacher-led initiative,
virtually all of the CAS hubs across Wales are led by academics
in University-based Technocamps Hubs
(Dr Tom Crick at Cardiff Metropolitan University,
Dr Helen Phillips at Cardiff University,
Prof Andrew Ware at University of South Wales,
Prof Vic Grout at Glynd\^{w}r University,
Prof Roger Boyle at Aberystwyth University,
and Dr Dave Perkins at Bangor University).
Teachers have generally not been self-motivated in Wales
to promote the CAS agenda.

An independent Review~\cite{Wavehill:2015}
of Technocamps activity in the convergence region of Wales
carried out for Welsh Government estimates that
5\% of Welsh secondary school-aged youths (ie, aged 11-19)
in the convergence area of Wales
have engaged with Technocamps through Workshops,
and that
more than a quarter of the secondary schools
in the region have established Technoclubs.
Furthermore, the new GCSE Computer Science curriculum
-- which remains optional for schools in Wales --
has now been adopted by a large percentage of these schools,
whilst schools outside of the convergence area
(and outside the reach of Technocamps) continue
to deliver the ICT curriculum.

Although it could not operate within
the non-convergence area of Wales,
Technocamps promoted all of its extensive on-line resources 
which are freely available to schools outside the convergence area of Wales,
and supported the activities of CAS Wales to develop the CAS model
of teacher-led in-school activities throughout Wales.
However, despite the efforts of CAS Wales,
there is scarce evidence of even a single active school-based
computing club that isn't inside the convergence area and
established due directly to Technocamps Workshops and follow-up engagement.

In support of this claim, consider the following.
The Annual Technocamps Robotics Competition was open to all Schools
across all of Wales, promoted across all of Wales
through Technocamps and CAS Wales channels, and even held
on the outskirts of Cardiff (in Treforest) in 2013.
However, every single one of the 43 teams entered in the 2013
competition held near Cardiff
travelled in from a convergence-area Technoclub formed on the back
of Technocamps Workshops and follow-up engagements with Technocamps.

This provides indisputable proof that the Technocamps model
of intense direct engagement through campus-based Workshops,
in conjunction with Teacher CPD and support, is crucial in order for success.
The lack of confidence and isolation felt by the teacher community
in Wales means that computing clubs have only arisen
-- and will continue to only arise --
through direct involvement of and engagement with Technocamps.


\section{Conclusions}
What are the top-line messages???
% data from ICT review survey?

% bib
\bibliographystyle{abbrv}
\bibliography{wipsce2015}

%\section*{To Do}
%
%In no particular order...
%
%\begin{itemize}
%\item Set UK context over past 3-5 years
%\item Link to English and Scottish changes
%\item Welsh context
%\item Technocamps: the ten year journey from 2003
%\item Convergence, the pan-Wales problem
%\item Key contributions: aims, targets, impacts, young people, NEETs, coverage
%\item Links with CAS Wales
%\item Hubs (both TC and CAS)
%\item Funding models: ESF, NSA, Nesta, etc
%\item {\textbf{Key theme}}: Building capacity, the problems of
%  England's NoE model in Wales
%\item UK policy: RS report, English curriculum, qualification change,
%  UKForCE, etc.
%\item Welsh policy: ICT curriculum, ICT review, Estyn reports, ICT
%  sector support/economic drivers, curriculum change,
%  Donaldson and post-Donaldson
%\item THE FUTURE...!
%\end{itemize}
%
%\subsection*{References to fit in}
%\begin{itemize}
%\item
%General CAS
%citations~\cite{crick+sentance:2011,brown-et-al-sigcse2012,brown-et-al-toce2014}.
%
%\item
%Teachers, CPD and
%NoE~\cite{sentance-et-al-wipsce2012,sentance-et-al:2013,sentance-et-al:2014}.
%
%\item
%Technocamps~\cite{ball-et-al:2012,boyle-et-al:2012}
%
%\item
%Welsh Government report:
%\begin{itemize}
%\item
%ICT Review~\cite{welshictreview:2013}
%\item
%Graham Report on STEM~\cite{STEMreview:2014}
%\item
%Donaldson Report~\cite{Donaldson:2015}
%\item
%Furlong Report~\cite{Furlong:2015}
%\end{itemize}
%
%\item
%Misc Reports
%\begin{itemize}
%\item
%NESTA Report~\cite{NESTA:2015}
%
%\end{itemize}
%\end{itemize}

\end{document}
